\documentclass[10pt]{article}

\usepackage[a4paper, landscape, margin=2.00cm]{geometry}
\usepackage{amsfonts, amssymb, amsmath}
\usepackage{graphicx}
\usepackage{float}
\usepackage{multirow}
\usepackage{array}
\usepackage[utf8]{inputenc}
\usepackage[T1]{fontenc}

\begin{document}
\renewcommand{\arraystretch}{1.50}

    \section*{git config}
    \begin{tabular}{|>{\tt}p{9.00cm}|>{}p{15.50cm}|}
        \hline
        git config -{}-global user.name <"name">                & postavljanje korisničkog imena \\ \hline
        git config -{}-global user.email <"email">              & postavljanje korisničkog emaila \\ \hline
        git config -{}-global core.editor "vim -{}-nofork"      & postavljanje Vim-a kao editora \\ \hline 
        git config -{}-global core.editor "atom -{}-wait"       & postavljanje Atom-a kao editora \\ \hline
        git config -{}-global core.editor "code -{}-wait"       & postavljanje VS Code-a kao editora \\ \hline
    \end{tabular}

    \section*{git init}
    \begin{tabular}{|>{\tt}p{9.00cm}|>{}p{15.50cm}|}
        \hline
        git init                                                & inicijalizacija trenutnog direktorija kao repozitorija \\ \hline
    \end{tabular}

    \section*{git clone}
    \begin{tabular}{|>{\tt}p{9.00cm}|>{}p{15.50cm}|}
        \hline
        git clone <URL>                                         & kloniranje repozitorija \\ \hline
    \end{tabular}

    \section*{git remote}
    \begin{tabular}{|>{\tt}p{9.00cm}|>{}p{15.50cm}|}
        \hline
        git remote [-v]                                 & [detaljni] prikaz \textit{remote}-ova \\ \hline
        git remote add upstream <URL>                   & dodavanje \textit{upstream remote}-a (za \textit{fetch}-anje \textit{update}-ova na originalnom repozitoriju) \\ \hline
        git remote add origin <URL>                     & dodavanje \textit{upstream remote}-a (za \textit{push}-anje \textit{update}-ova na \textit{forked} repozitorij) \\ \hline
        git remote rename <old> <"new">                 & preimenovanje \textit{remote}-a \\ \hline  
        git remote remove <name>                        & uklanjanje \textit{remote}-a \\ \hline                 
    \end{tabular}

    \section*{git fetch}
    \begin{tabular}{|>{\tt}p{9.00cm}|>{}p{15.50cm}|}
        \hline
        git fetch <remote>                              & \textit{update}-a stanje svih \textit{branch}-eva \textit{remote} repozitorija \\ \hline
        git fetch <remote> <remote-branch>              & \textit{update}-a stanje pojedinačnog \textit{branch}-a \textit{remote} repozitorija \\ \hline                
    \end{tabular}

    \section*{git pull}
    \begin{itemize}
        \item \texttt{git pull origin <remote-branch> = git fetch origin <remote-branch> + git merge <remote>/<branch>}
    \end{itemize}
    \begin{tabular}{|>{\tt}p{9.00cm}|>{}p{15.50cm}|}
        \hline
        git pull <remote> <remote-branch>               & \textit{update}-a stanje pojedinačnog \textit{branch}-a remote repozitorija i sukladno \textit{update}-a lokalno stanje \\ \hline                
    \end{tabular}

    \section*{git push}
    \begin{tabular}{|>{\tt}p{9.00cm}|>{}p{15.50cm}|}
        \hline
        git push <remote> <local-branch>                    & \textit{push}-anje lokalnog \textit{branch}-a na \textit{remote} istog naziva \\ \hline
        git push -{}-tags <remote> <local-branch>           & \textit{push}-a tagove na remote repozitorij (\textit{tag}-ovi se neće \textit{push}-ati automatski) \\ \hline                   
    \end{tabular}

    \section*{git status}
    \begin{tabular}{|>{\tt}p{9.00cm}|>{}p{15.50cm}|}
        \hline
        git status                                          & status repozitorija (\textit{branch}, \textit{untracked}, \textit{unstaged}, \textit{staged}\dots) \\ \hline
    \end{tabular}

    \section*{git log}
    \begin{tabular}{|>{\tt}p{9.00cm}|>{}p{15.50cm}|}
        \hline
        git log                                             & puni ispis svih \textit{commit}-a na repozitoriju \\ \hline 
        git log -{}-oneline                                 & jednolinijski ispis svih \textit{commit}-a na repozitoriju \\ \hline
    \end{tabular}

    \section*{git add}
    \begin{tabular}{|>{\tt}p{9.00cm}|>{}p{15.50cm}|}
        \hline
        git add <file>                                      & dodavanje datoteka u \textit{staging area} \\ \hline
    \end{tabular}

    \section*{git commit}
    \begin{tabular}{|>{\tt}p{9.00cm}|>{}p{15.50cm}|}
        \hline
        git commit                                          & \textit{commit} \textit{staged} datoteka (naknadni unos \textit{commit} poruke) \\ \hline
        git commit -m <message>                             & \textit{commit} \textit{staged} datoteka uz poruku \\ \hline
        git commit -a -m <message>                          & dodavanje svog sadržaja radnog direktorija u staging area i \textit{commit} uz poruku \\ \hline
        git commit -{}-amend                                & \textit{redo} posljednjeg \textit{commit}-a ili uređivanje poruke posljednjeg \textit{commit}-a \\ \hline
    \end{tabular}

    \section*{git branch}
    \begin{tabular}{|>{\tt}p{9.00cm}|>{}p{15.50cm}|}
        \hline
        git branch -l [-v]                              & [detaljni] prikaz lokalnih \textit{branch}-eva \\ \hline
        git branch -r [-v]                              & [detaljni] prikaz remote \textit{branch}-eva \\ \hline
        git branch -a [-v]                              & [detaljni] prikaz svih \textit{branch}-eva \\ \hline
        git branch <branch-name>                        & kreiranje novog \textit{branch}-a \\ \hline
        git branch -m <branch-name>                     & preimenovanje aktivnog \textit{branch}-a (koristiti \texttt{-M} za nasilno) \\ \hline
        git branch -d <branch-name>                     & brisanje \textit{branch}-a (koristiti \texttt{-D} za nasilno) \\ \hline
    \end{tabular}

    \section*{git switch}
    \begin{tabular}{|>{\tt}p{9.00cm}|>{}p{15.50cm}|}
        \hline
        git switch <branch-name>                        & prebacivanje s trenutnog na drugi \textit{branch} (ako se koristi ime \textit{remote branch}-a, kreira lokalnu kopiju) \\ \hline
        git switch -                                    & prebacivanje u prethodno posjećeni \textit{branch} \\ \hline
        git switch -c <branch-name>                     & kreiranje novog \textit{branch}-a s imenom name s trenutnog \textit{HEAD}-a i prebacivanje u njega \\ \hline
    \end{tabular}

    \section*{git merge}
    \begin{itemize}
        \item \textit{merge}-ani \textit{branch} se ne briše
    \end{itemize}
    \begin{tabular}{|>{\tt}p{9.00cm}|>{}p{15.50cm}|}
        \hline
        git merge <branch-name>                         & \textit{merge}-anje \textit{branch}-a u aktivni \textit{branch} \\ \hline
    \end{tabular}

    \section*{git rebase}
    \begin{itemize}
        \item rebase funkcijski može služiti kao zamjena za \textit{merge}
        \begin{itemize}
            \item reorganizira/kopira \textit{commit} s aktualnog \textit{branch}-a na vrh master \textit{branch}-a
        \end{itemize}
    \end{itemize}
    \begin{tabular}{|>{\tt}p{9.00cm}|>{}p{15.50cm}|}
        \hline
        git rebase main                                 & reorganizacija/kopiranje \textit{commit}-a s aktualnog \textit{branch}-a na vrh main \textit{branch}-a \\ \hline
        git rebase -i <commit-hash>                     & ulazak u interkativno uređivanje prethodnih \textit{commit}-a, počevši od \textit{commit-hash}-a \\ \hline
    \end{tabular}

    \section*{git diff}
    \begin{tabular}{|>{\tt}p{9.00cm}|>{}p{15.50cm}|}
        \hline
        git diff <file>                                         & prikazivanje \textit{unstaged} promjena datoteke \\ \hline
        git diff -{}-staged <file>                              & prikazivanje \textit{staged} promjena datoteke \\ \hline
        git diff HEAD <file>                                    & prikazivanje svih promjena datoteke od posljednjeg \textit{commit}-a \\ \hline
        git diff <old-commit-hash> <new-commit-hash>            & prikazivanje promjena između dva \textit{commit}-a \\ \hline
        git diff <branch-name-1> <branch-name-2>                & prikazivanje promjena između "vrhova" \textit{branch}-eva \\ \hline
        git diff <branch-name>@\{0\} <branch-name>@\{1\}        & prikazuje promjene između stanja na \textit{branch}-u (vidi \texttt{git reflog show}) \\ \hline
    \end{tabular}

    \section*{git stash}
    \begin{itemize}
        \item \textit{stash} je globalan, nije vezan za \textit{branch}
        \begin{itemize}
            \item moguće je aplicirati \textit{stashed} promjene s jednog na drugi \textit{branch}
        \end{itemize}
    \end{itemize}
    \begin{tabular}{|>{\tt}p{9.00cm}|>{}p{15.50cm}|}
        \hline
        git stash save                                  & spremanje \textit{unstaged} i \textit{staged} promjena u \textit{stash} \\ \hline
        git stash list                                  & listanje \textit{stashed} promjena \\ \hline
        git stash pop [stash@{<number>}]                & apliciranje \textit{stashed} promjena na radni direktorij (cut iz \textit{stash}-a) \\ \hline
        git stash apply [stash@{<number>}]              & apliciranje \textit{stashed} promjena na radni direktorij (copy iz \textit{stash}-a) \\ \hline
        git stash drop [stash@{<number>}]               & brisanje \textit{stashed} promjene \\ \hline
        git stash clear                                 & brisanje svih \textit{stashed} promjena iz liste \\ \hline
    \end{tabular}

    \section*{git checkout}
    \begin{tabular}{|>{\tt}p{9.00cm}|>{}p{15.50cm}|}
        \hline
        git checkout <branch-name>                      & prebacivanje s trenutnog \textit{HEAD}-a na drugi \textit{branch} (funkcionira i za \textit{remote branch}-eve) \\ \hline
        git checkout -b <branch-name>                   & kreiranje novog \textit{branch}-a s imenom name s trenutnog \textit{HEAD}-a i prebacivanje u njega \\ \hline
        git checkout -{}-track <remote>/<branch>        & kreira lokalnu kopiju \textit{remote branch}-a istog naziva (vidi \texttt{git switch}) \\ \hline
        git checkout <commit-hash>                      & prebacivanje \textit{HEAD}-a na neki drugi \textit{commit} \\ \hline
        git checkout <tag>                              & prebacivanje \textit{HEAD}-a na neki drugi \textit{commit} označen \textit{tag}-om \\ \hline
        git checkout HEAD <file>                        & poništavanje svih promjena do posljednjeg \textit{commit}-a za datoteku \\ \hline
        git checkout HEAD@\{<number>\}                  & premještanje u poziciju \textit{HEAD}-a prije \textit{number} koraka (vidi \texttt{git reflog show}) \\ \hline
        git checkout <branch-name>@\{<number>\}         & premještanje u stanje \textit{branch}-a prije \textit{number} koraka (vidi \texttt{git reflog show}) \\ \hline
    \end{tabular}

    \section*{git tag}
    \begin{itemize}
        \item \textit{tag} je dodatan pointer na \textit{commit}
        \begin{itemize}
            \item \textit{tag}-ovi su jedinstveni
        \end{itemize}
        \item uglavnom služe za \textit{semantic versioning}
    \end{itemize}
    \begin{tabular}{|>{\tt}p{9.00cm}|>{}p{15.50cm}|}
        \hline
        git tag -l ["<regex-search>"]                   & ispisuje sve \textit{tag}-ove na aktualnom \textit{branch}-u [koji zadovoljavaju \textit{regex} pretragu] \\ \hline
        git tag [-a] <"tag-name">                       & tagira aktualni \textit{HEAD}/\textit{commit} s [\textit{annotated}] \textit{tag}-om \\ \hline
        git tag [-a] <"tag-name"> <commit-hash>         & tagira \textit{commit} s [\textit{annotated}] \textit{tag}-om \\ \hline
        git tag -f <"tag-name"> <commit-hash>           & "nasilno" tagira \textit{commit} s već iskorištenim \textit{tag}-om (tag sa starog \textit{commit}-a nestaje) \\ \hline
        git tag -d <tag-name>                           & briše \textit{tag} \\ \hline
    \end{tabular}

    \section*{git restore}
    \begin{itemize}
        \item nikad ne uklanja \textit{commit log}
    \end{itemize}
    \begin{tabular}{|>{\tt}p{9.00cm}|>{}p{15.50cm}|}
        \hline
        git restore -{}-staged <file>                   & prebacivanje \textit{staged} promjena u \textit{unstaged} \\ \hline
        git restore <file>                              & poništavanje \textit{unstaged} promjena do posljednjeg \textit{commit}-a \\ \hline
        git restore -{}-source=<commit-hash> <file>     & poništavanje promjena po uzoru na \textit{commit} (\textit{log} ostaje isti, promjene odlaze u \textit{unstaged}) \\ \hline
    \end{tabular}

    \section*{git reset}
    \begin{itemize}
        \item uvijek uklanja \textit{commit log}
        \item \textit{unstaged} i \textit{staged} promjene se mogu prebaciti na drugi branch
    \end{itemize}
    \begin{tabular}{|>{\tt}p{9.00cm}|>{}p{15.50cm}|}
        \hline
        git reset -{}-soft <commit-hash>                & poništavanje promjena po uzoru na \textit{commit} (\textit{log} se mijenja, promjene odlaze u \textit{staged}) \\ \hline
        git reset [-{}-mixed] <commit-hash>             & poništavanje promjena po uzoru na \textit{commit} (\textit{log} se mijenja, promjene odlaze u \textit{unstaged}) \\ \hline
        git reset -{}-hard <commit-hash>                & poništavanje promjena po uzoru na \textit{commit} (\textit{log} se mijenja, promjene se \textit{commit}-aju) \\ \hline
    \end{tabular}

    \section*{git revert}
    \begin{itemize}
        \item nikad ne uklanja \textit{commit log}
    \end{itemize}
    \begin{tabular}{|>{\tt}p{9.00cm}|>{}p{15.50cm}|}
        \hline
        git revert <commit-hash>                        & poništavanje promjena od \textit{commit}-a i dalje (\textit{log} ostaje isti, promjene odlaze u \textit{unstaged-conflicted}) \\ \hline
    \end{tabular}

    \section*{git reflog}
    \begin{itemize}
        \item \textit{reflog}-ovi su lokalni i imaju "rok trajanja"
        \item služe za poništavanje \texttt{git reset} i sl.
    \end{itemize}
    \begin{tabular}{|>{\tt}p{9.00cm}|>{}p{15.50cm}|}
        \hline
        git reflog show HEAD                        & prikazuje povijest "kretanja" \textit{HEAD}-a | povijest radnji na \textit{branch}-u \\ \hline
        git reflog show <branch-name>               & prikazuje povijest stanja na \textit{branch}-u \\ \hline
        git reflog show HEAD@{2.day.ago}            & prikazuje povijest "kretanja" \textit{HEAD}-a do prije 2 dana \\ \hline
        git reflog show <branch-name>@{2.day.ago}   & prikazuje povijest stanja na \textit{branch}-u do prije 2 dana \\ \hline
    \end{tabular}

\end{document}